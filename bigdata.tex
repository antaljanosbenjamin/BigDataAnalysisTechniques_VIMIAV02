\documentclass{article}

\usepackage[utf8]{inputenc} 

\usepackage{hyperref}
\hypersetup{
    colorlinks=false,
    pdfborder={0 0 0},
}

\usepackage{hhline}

\usepackage{geometry}
\geometry{ margin=1in}

\usepackage{graphicx}
\graphicspath{ {pyspark_code/} }

\usepackage{amsmath}
\numberwithin{figure}{section}
\numberwithin{table}{section}

\newenvironment{compactlist}
{ \begin{itemize}
    \setlength{\itemsep}{0pt}
    \setlength{\parskip}{0pt}
    \setlength{\parsep}{0pt}     }
{ \end{itemize}} 

\newcommand{\secref}[1]{\aref{sec:#1}.~fejezet}
\newcommand{\figref}[1]{\aref{fig:#1}.~ábr}
\newcommand{\tabref}[1]{\aref{tab:#1}.~táblázat}
\newcommand{\Secref}[1]{\Aref{sec:#1}.~fejezet}
\newcommand{\Figref}[1]{\Aref{fig:#1}.~ábr}
\newcommand{\Tabref}[1]{\Aref{tab:#1}.~táblázat}

\usepackage[magyar]{babel}

\usepackage{multicol}
\setlength{\columnsep}{1cm}

\date{2016-11-27}
\title{'Big Data' elemzési módszerek házi feladat}

\author{Antal János Benjamin\\G9PTHG \and Dovala Rozina\\Z35SBZ \and Gerendai Zsuzsa\\FQPRHP}

\begin{document}
	\begin{sloppypar}
		\pagenumbering{gobble}
		\maketitle
		\newpage
		\tableofcontents
		\newpage
		\pagenumbering{arabic}
				\section{Bevezetés}
		\subsection{Adatforrás}
		\paragraph{}	
		A házi feladatunk Ausztrália 18485 egységből álló, publikus használatban lévő mosdóinak az elemzése volt. Ehhez rendelkezésünkre állt egy állandóan frissülő .csv fájl, amely ezeknek a mosdóknak a különböző tulajdonságait tartalmazza.\par		
		Forrásaink a feladatkiosztásnál megjelölt \href{https://data.gov.au/dataset/national-public-toilet-map}{honlap}, illetve a mosdók állam által üzemeltetett közös \href{https://toiletmap.gov.au}{honlapja} volt. Az összefoglaló .csv fájlban a mosdók adatairól olvashattunk, míg az egyes attribútumok jelentését az egységes honlap segítségével társítottuk.
		\subsection{Attribútumok}
	\begin{compactlist}
		\item \textbf{Tulajdonság neve} - \textit{Típusa} : Jelentése (Megjegyzés, ha van)
		\item \textbf{Toilet ID} - \textit{Egész szám} : a mosdó azonosítója (1-től indul a számozás)
		\item \textbf{URL} - \textit{Szöveg} : a mosdó egyedi honlapja 
		\item \textbf{Name} - \textit{Szöveg} : a hely vagy intézmény neve, ahol a mosdó található
		\item \textbf{Address1} - \textit{Szöveg} : közterületi egység (sugárút/út/utca stb.)
		\item \textbf{Town} - \textit{Szöveg} : város
		\item \textbf{State} - \textit{Szöveg} : térség
		\item \textbf{Postcode} - \textit{Szöveg} : irányítószám
		\item \textbf{AddressNote} - \textit{Szöveg} : megjegyzések az elhelyezkedéssel kapcsolatosan (például a mosdó a park észak-nyugati csücskében van)
		\item \textbf{Male} - \textit{Logikai} : férfi mosdó
		\item \textbf{Female} - \textit{Logikai} : női mosdó
		\item \textbf{Unisex} - \textit{Logikai} : uniszex mosdó
		\item \textbf{DumpPoint} - \textit{Logikai} : ürítési pont (szemét, lakókocsiban felgyülemlett vizelet stb. elhelyezésére)
		\item \textbf{FacilityType} - \textit{Szöveg} : adódó lehetőségek, ahol a mosdó helyezkedik (például buszmegálló, park vagy sportpálya)
		\item \textbf{ToiletType} - \textit{Szöveg} : mosdó típusa (például szennyvízelvezetéses vagy sem)
		\item \textbf{AccessLimited} - \textit{Logikai} : korlátolt hozzáférés
		\item \textbf{PaymentRequired} - \textit{Logikai} : kell-e érte fizetni
		\item \textbf{KeyRequired} - \textit{Logikai} : kulcsot kell-e kérni
		\item \textbf{AccessNote} - \textit{Szöveg} : megjegyzések a hozzáféréssel kapcsolatosan (például a mosdó egy bevásárlóközpontban van)
		\item \textbf{Parking} - \textit{Logikai} : parkolási lehetőség
		\item \textbf{ParkingNote} - \textit{Szöveg} : parkolással kapcsolatos megjegyzések
		\item \textbf{AccessibleMale} - \textit{Logikai} : férfi mozgássérült mosdó
		\item \textbf{AccessibleFemale} - \textit{Logikai} : női mozgássérült mosdó
		\item \textbf{AccessibleUnisex} - \textit{Logikai} : uniszex mozgássérült mosdó
		\item \textbf{AccessibleNote} - \textit{Szöveg} : megjegyzés a mozgássérült mosdókkal kapcsolatban (például kerekes székkel csak az egyik irányból lehet megközelíteni)
		\item \textbf{MLAK} - \textit{Logikai} : MLAK kulcs használható-e (MLAK = Master Locksmiths' Association Key, egy mesterkulcs a publikus mosdók nyitásához, amelyet külön lehet igényelni orvosi engedéllyel)
		\item \textbf{ParkingAccessible} - \textit{Logikai} : mozgássérült parkoló lehetőség
		\item \textbf{AccessibleParkingNote} - \textit{Szöveg} : mozgássérült parkolási megjegyzések (például csak az utcafronton lehet parkolni)
		\item \textbf{Ambulant} - \textit{Logikai} : ambuláns betegeknek használható-e
		\item \textbf{RHTransfer} - \textit{Logikai} : jobbkezes nyitású ajtó mozgássérülteknek
		\item \textbf{LHTransfer} - \textit{Logikai} : balkezes nyitású ajtó mozgássérülteknek
		\item \textbf{AdultChange} - \textit{Logikai} : felnőtt pelenkázó
		\item \textbf{OpeningHoursSchedule} - \textit{Szöveg} : nyitvatartási idő
		\item \textbf{OpeningHoursNote} - \textit{Szöveg} : nyitvartartáshoz megjegyzések
		\item \textbf{BabyChange} - \textit{Logikai} : gyermek pelenkázó
		\item \textbf{Showers} - \textit{Logikai} : zuhanyzó
		\item \textbf{DrinkingWater} - \textit{Logikai} : ivóvíz lehetősége
		\item \textbf{SharpsDisposal} - \textit{Logikai} : gyógyszer- és vegyi anyag tároló
		\item \textbf{SanitaryDisposal} - \textit{Logikai} : egészségügyi hulladék tároló
		\item \textbf{IconURL} - \textit{Szöveg} : jelölő ikon URL-je
		\item \textbf{IconAltText} - \textit{Szöveg} : szövegezés (milyen mosdólehetőségek vannak, röviden, például női-férfi-uniszex)
		\item \textbf{Notes} - \textit{Szöveg} : egyéb megjegyzések (például van a közelben BBQ étterem és lehetőség)
		\item \textbf{Status} - \textit{Szöveg} : Hitelesített vagy sem (Verified/Unverified) (hitelesített vagy nem)
		\item \textbf{Latitude} - \textit{Lebegőpontos szám} : szélességi fok
		\item \textbf{Longitude} - \textit{Lebegőpontos szám} : hosszúsági fok
	\end{compactlist}
	\paragraph{}
	Néhány szöveges mezőhöz valójában kategorikus változónak feleltethető meg. A két legfontosabb ezek közül a \textit{FacilityType} és a \textit{ToiletType}. A \textit{FacilityType} a következő értékeket veszi fel:
	\begin{compactlist}
		\item Airport
		\item Bus station
		\item Camping ground
		\item Car park
		\item Caravan park
		\item Food outlet
		\item Other
		\item Park or reserve
		\item Service station
		\item Shopping centre
		\item Sporting facility
		\item Train station
		\item Üres
	\end{compactlist}
	A \textit{ToiletType} pedig az alábbiakat:
	\begin{compactlist}
		\item Automatic
		\item Compost
		\item Drop toilet
		\item Pit
		\item Sealed Vault
		\item Septic
		\item Sewerage
		\item Üres
	\end{compactlist}
	
		\newpage
			\section{Felderítő adatelemzés}
	\subsection{Általános adatok}
	\paragraph{}
	Az adathalmaz több szöveges és logikai mezőt tartalmaz. A szöveges mezőkkel egy kivételével a felderítő adatelemzés folyamán nem foglalkoztunk. A \textit{Status} mezőből készítettünk egy \textit{StatusNum} nevű mezőt, ahol 1 jelenti a \textit{Verified} és 0 a \textit{Unverified} értéket. \par
	A logikai mezőkhöz felvettünk két tovább mezőt az alábbi jelentésekkel: \textit{RealUnisexNum = (Male AND Female) OR Unisex}, és \textit{RealAccessibleUnisexNum = (AccessibleMale AND AccessibleFemale) OR AccessibleUnisex}. Ezután minden logikai mezőhöz készítettünk egy egész szám típusú mezőt, ahol 0 jelenti a hamis, 1 pedig az igazi értéket. Végül a koordinátákat tartalmazó és az előállított számértékű mezőkre megállapítottuk az alapvető statisztikai jellemzőket:
	\begin{table}[!ht]
		\centering
		\begin{tabular}{ | l | l | l | l | l | l | }
			\hline
			\textbf{Mezőnév} & \textbf{Átlag} & \textbf{Szórás} & \textbf{Minimum} & \textbf{Maximum}\\ \hline
			Latitude & -32.6784 & 5.4337 & -43.582 & -10.5702\\ \hline
			Longitude & 144.1794 & 10.7652 & 113.4102 & 153.6263\\ \hline
			MaleNum & 0.8048 & 0.3964 & 0 & 1\\ \hline
			FemaleNum & 0.8039 & 0.3971 & 0 & 1\\ \hline
			UnisexNum & 0.1233 & 0.3288 & 0 & 1\\ \hline
			DumpPointNum & 0.0277 & 0.1641 & 0 & 1\\ \hline
			AccessLimitedNum & 0.037 & 0.1888 & 0 & 1\\ \hline
			PaymentRequiredNum & 0.0085 & 0.0921 & 0 & 1\\ \hline
			KeyRequiredNum & 0.023 & 0.1499 & 0 & 1\\ \hline
			ParkingNum & 0.3569 & 0.4791 & 0 & 1\\ \hline
			AccessibleMaleNum & 0.223 & 0.4163 & 0 & 1\\ \hline
			AccessibleFemaleNum & 0.2237 & 0.4168 & 0 & 1\\ \hline
			AccessibleUnisexNum & 0.2751 & 0.4466 & 0 & 1\\ \hline
			MLAKNum & 0.0281 & 0.1652 & 0 & 1\\ \hline
			ParkingAccessibleNum & 0.1922 & 0.394 & 0 & 1\\ \hline
			AmbulantNum & 0.0113 & 0.1055 & 0 & 1\\ \hline
			LHTransferNum & 0.0323 & 0.1768 & 0 & 1\\ \hline
			RHTransferNum & 0.0373 & 0.1896 & 0 & 1\\ \hline
			AdultChangeNum & 0.0021 & 0.0459 & 0 & 1\\ \hline
			BabyChangeNum & 0.1099 & 0.3128 & 0 & 1\\ \hline
			ShowersNum & 0.0563 & 0.2304 & 0 & 1\\ \hline
			DrinkingWaterNum & 0.072 & 0.2584 & 0 & 1\\ \hline
			SharpsDisposalNum & 0.1477 & 0.3548 & 0 & 1\\ \hline
			StatusNum & 0.7748 & 0.4177 & 0 & 1\\ \hline
			RealUnisexNum & 0.9245 & 0.2641 & 0 & 1\\ \hline
			RealAccessibleUnisexNum & 0.4974 & 0.5 & 0 & 1\\ \hline
		\end{tabular}
		\caption{Általános statisztikai leírók}
	\end{table}\par
	A logikai értékekhez előállított mezőknél az átlag azt jelenti, hogy az adatsorok mekkora százalékában vesz fel igaz értéket az adott mező. Ebből látható, hogy a mosdók 92\%-ban a nők és a férfiak által használatba vehetőek, 80\%-ban pedig van külön férfi és női mosdó is. Mozgássérült mosdó az esetek felében érhető el, melyek felében unisex, másik felében külön férfi és női mosdó van a mozgássérülteknek.\par	
	Az általános statisztikai leírók után megvizsgáltuk a 0.5-nél nagyobb abszolútértékű korrelációval rendelkező mezőpárokat, melyek az alábbiak:
	\begin{compactlist}			
		\item $corr(AccessibleFemaleNum, AccessibleMaleNum) = 0.9939$
		\item $corr(FemaleNum, MaleNum) = 0.9799$
		\item $corr(MaleNum, UnisexNum) = -0.7617$
		\item $corr(FemaleNum, UnisexNum) = -0.7594$
		\item $corr(AccessibleUnisexNum, RealAccessibleUnisexNum) = 0.6192$
		\item $corr(FemaleNum, RealUnisexNum) = 0.5527$
		\item $corr(MaleNum, RealUnisexNum) = 0.5455$
		\item $corr(AccessibleMaleNum, RealAccessibleUnisexNum) = 0.5352$
		\item $corr(AccessibleFemaleNum, RealAccessibleUnisexNum) = 0.5329$
		\item $corr(ParkingAccessibleNum, ParkingNum) = 0.5312$
	\end{compactlist}\par
	A korrelációk alapján a következő sejtéseink vannak:
	\begin{compactlist}			
		\item A férfi és a női mosdók általában együtt jelennek meg, azaz kevés olyan hely van, ahol csak az egyik van. Ugyanez igaz a mozgássérült női-férfi mosdókra.
		\item Ha van férfi vagy női mosdó, akkor nagy eséllyel nincs unisex mosdó.
		\item A mozgássérült mosdók nagyjából fele csak unisex, másik fele pedig külön női-férfi mosdóval rendelkező mozgássérült mosdó.
		\item A parkolók nagyrészében van mozgássérült parkoló.
	\end{compactlist}
	A többi korreláció bár nem elhanyagolhatók, de sejtések megfogalmazásához nem elég nagyok.
	\subsection{Mosdók eloszlása}
	\paragraph{}
	A mosdókat scatterplot diagramon ábrázolva megláthatjuk, a mosdók az óceán közelében sűrűsödnek.
	\begin{figure}[!ht]
		\centering
			\includegraphics[scale=0.7]{scatter}
			\caption{Nyílvános mosdók eloszlása Ausztráliában}
	\end{figure}
	\paragraph{}
	A diagramon megfigyelhető továbbá, hogy a nagyvárosok közelében a mosdók annyira sűrűn helyezkednek el, hogy a scatterploton egy egységes kék területet kapunk. a jobb áttekinthetőség érdekében ezért heatmap-en is ábrázoltuk a mosdókat.
	\begin{figure}[!ht]
		\centering	
			\includegraphics[scale=0.5]{heat_without_max}
			\caption{Nyílvános mosdók Ausztráliában, heatmap-en ábrázolva}
	\end{figure}
	\paragraph{}
	Ekkor azonban látszódik hogy a két szélsőséges eset, ahol közel 800 mosdó kerül egy bin-be elnyomja a többi értéket. Az ábrán azonban jól látszik, hogy a mosdók sűrűsége az óceánhoz közel magasabb, mint a sziget belsejében. A jobb áttekinthetőség érdekében a túl magas értékkel rendelkező bineket maximalizáltuk, továbbá a bin-ek számát is megváltoztattuk. A próbálkozások a dokumentum végén megtekinthetőek, a szerintünk legjobbnak gondolt diagram a 2.3 ábrán látható.	
	\begin{figure}[!ht]
		\centering	
			\includegraphics[scale=0.5]{heat_4_30}
			\caption{Nyílvános mosdók Ausztráliában, heatmap-en ábrázolva, megfelelő bin számmal és maximum értékkel}
	\end{figure}
	\clearpage
	\subsection{Heatmap variációk}
	\paragraph{}
	Az itt látható heatmapek esetén 35x45 bint használtunk, a maximum érték pedig 10,20,30,40 vagy 50. 
	\begin{figure}[h]
		\begin{multicols}{2}
			\includegraphics[scale=0.35]{heat_1_10}
			\includegraphics[scale=0.35]{heat_1_20}
			\includegraphics[scale=0.35]{heat_1_30}
			\includegraphics[scale=0.35]{heat_1_40}
			\includegraphics[scale=0.35]{heat_1_50}		
		\end{multicols}
	\end{figure}
	\clearpage	
	\paragraph{}
	Az itt látható heatmapek esetén 70x90 bint használtunk, a maximum érték pedig szintén 10,20,30,40 vagy 50. 
	\begin{figure}[h]
		\begin{multicols}{2}
			\includegraphics[scale=0.35]{heat_2_10}
			\includegraphics[scale=0.35]{heat_2_20}
			\includegraphics[scale=0.35]{heat_2_30}
			\includegraphics[scale=0.35]{heat_2_40}
			\includegraphics[scale=0.35]{heat_2_50}		
		\end{multicols}
	\end{figure}	
	\clearpage	
	\paragraph{}
	Az itt látható heatmapek esetén 105x135 bint használtunk, a maximum érték pedig szintén 10,20,30,40 vagy 50. 
	\begin{figure}[h]
		\begin{multicols}{2}
			\includegraphics[scale=0.35]{heat_3_10}
			\includegraphics[scale=0.35]{heat_3_20}
			\includegraphics[scale=0.35]{heat_3_30}
			\includegraphics[scale=0.35]{heat_3_40}
			\includegraphics[scale=0.35]{heat_3_50}		
		\end{multicols}
	\end{figure}	
	\clearpage	
	\paragraph{}
	Az itt látható heatmapek esetén 140x180 bint használtunk, a maximum érték pedig szintén 10,20,30,40 vagy 50. 
	\begin{figure}[h]
		\begin{multicols}{2}
			\includegraphics[scale=0.35]{heat_4_10}
			\includegraphics[scale=0.35]{heat_4_20}
			\includegraphics[scale=0.35]{heat_4_30}
			\includegraphics[scale=0.35]{heat_4_40}
			\includegraphics[scale=0.35]{heat_4_50}		
		\end{multicols}
	\end{figure}	
	\clearpage	
	\paragraph{}
	Az itt látható heatmapek esetén 1750x225 bint használtunk, a maximum érték pedig szintén 10,20,30,40 vagy 50. 
	\begin{figure}[h]
		\begin{multicols}{2}
			\includegraphics[scale=0.35]{heat_5_10}
			\includegraphics[scale=0.35]{heat_5_20}
			\includegraphics[scale=0.35]{heat_5_30}
			\includegraphics[scale=0.35]{heat_5_40}
			\includegraphics[scale=0.35]{heat_5_50}		
		\end{multicols}
	\end{figure}
		\newpage
			\section{Hipotézisek}
	\paragraph{}
	Az adatok további vizsgálatai után a felderítő adatelemzés sejtéseit kiegészítve az alábbi hipotéziseket állítottuk fel. 
	\subsection{Nemek szerinti vizsgálat}
	\paragraph{}
	A felderítő adatelemzés során kiemelt korrelációk alapján két hipotézist állítottunk fel:
	\begin{compactlist}			
		\item Ha van női vagy férfi mosdó, akkor nagy valószínűséggel nincs unisex mosdó. Ugyanez igaz a mozgáskorlátozott típusú mosdóknál is.
		\item Vélhetően kevés esetben van női/férfi mosdó férfi/női mosdó nélkül. Ugyanez igaz a mozgáskorlátozott típusú mosdóknál is.
	\end{compactlist}
	Ezeket a hipotéziseket a megfelelő típusú rekordok szűrésével és aggregációjával kívánjuk igazolni.\par
	\subsection{Bevásárlóközpontok mosdói}
	\paragraph{}
	Ha a létesítmény típusa \textit{Shopping Center}, akkor az valószínűleg a parthoz közel, a sűrűn lakott területen helyezkedik el és nem sivatag közepén. Ennek a hipotézisnek a bizonyításához a létesítmények típusát kell elsősorban vizsgálni és ez alapján ábrázolni a mosdókat.	
	\subsection{Mosdók sűrűsége}
	\paragraph{}
	A heatmap alapján arra a következtetésre jutottunk, hogy a mosdók geolokációs sűrűsége feltehetően követi a népsűrűséget. A hipotézis igazlásához szükség lesz Ausztrália népsűrűségi adataira. Ezt a hipotézist az államok, esetleg városok népességi adataival történő összehasonlító elemzéssel kívánjuk igazolni.
	\subsection{Egyéb feltevések}
	\paragraph{}
	Több más hipotézist is megvizsgáltunk, amelyek ugyan nem bírnak nagy jelentőséggel, de emellett a kapott eredmények sem lettek meghatározóak.\par
	Egy ilyen feltevésünk volt, hogy a mozgássérült mosdók jelentős részénél használhatjuk az erre a célra létrehozott MLAK kulcsot. Az EDA során fény derült arra, hogy MLAK kulcsot a mosdóknak csupán a 2,8\%-ánál lehet használni. Ezt az értéket vetítettük arra, hogy a mozgássérült mosdók mekkora részében van jelen az MLAK: esetünkben ez kb. 5\% lett, amely szintén nem számottevő.\par
	Vizsgálatunk tárgyát képezte az is, hogy vajon van-e olyan mosdó, amelynél uniszex mosdó nélkül csak női vagy csak férfi mosdó található. Létezik egy-két ilyen eset, ellenben ez sem számottevő, az előzőekben láthattuk, hogy meghatározóan a női/férfi illemhelyet követi a párja is, ezek pedig az esetek kb. felében kiegészülnek mozgássérült lehetőséggel is.\par
	A nyilvántartott mosdókat csoportokba rendeztük típusuk szerint is (ToiletType mező értékei), amely esetben sajnos meghatározó mennyiséghez nem tartozott kitöltött mezőérték, így nagyjából csak az adathalmaz 45\%-ról kaptunk információt. A maradék mosdók 72\%-a Sewerage típusú, 16\%-a Septic típusú. A maradék 12\%-on 5 típus osztozik: Compost, Sealed Vault, Pit, Drop toilet, Automatic. A két legnagyobb csoportot  szűrve is ábrázoltuk a maradék rekordokat a jobb láthatóság érdekében. Érdekes megfigyelés, hogy \figref{without_two_biggest}án a "sivatag közepén" van egy \textit{Automatic} típusú mosdó, ami a környezettől eléggé szokatlan. Rövid kutatás után kiderült, hogy ott van \textit{Alice Springs}, egy nagy gépek fogadására is alkalmas reptérrel rendelkező kisváros.
	\begin{compactlist}
		\item Septic = kék
		\item Compost = zöld
		\item Sewerage = citromsárga
		\item Sealed Vault = fekete
		\item Pit = ciánkék
		\item Drop toilet = piros
		\item Automatic = magenta
	\end{compactlist}\par
	\begin{figure}[!ht]
		\centering	
			\includegraphics[scale=0.5]{type_of_toilets}
			\caption{A mosdók típusainak ábrázolása, az összes típus}
	\end{figure}
	\begin{figure}[!ht]
		\centering	
			\includegraphics[scale=0.5]{type_of_toilets_without_sew}
			\caption{A mosdók típusainak ábrázolása, a \textit{Sewerage} típusuak nélkül}
	\end{figure}
	\begin{figure}[!ht]
		\centering	
			\includegraphics[scale=0.5]{type_of_toilets_without_sew_and_sep}
			\caption{A mosdók típusainak ábrázolása, a \textit{Sewerage} és \textit{Septic} típusuak nélkül}
			\label{fig:without_two_biggest}
	\end{figure}
	Igyekeztünk valamilyen érdekességet találni a FacilityType attribútum alapján is, amely a mosdókhoz kapcsolt létesítmények lehetőségeiről ad plusz információt, például repülőtér, busz- vagy vonatállomás, park stb. van a közelben, vagy ott van a mosdó. Egy ilyen érték a ‘Caravan park’, amely megragadta a figyelmünket, így a lakókocsik révén megvizsgáltuk, hogy ezeknél a létesítményeknél milyen valószínűséggel találunk ürítési pontot (DumpPoint-ot). Szintén nem kaptunk értékelhető eredményeket, csupán 14\%-ot, mivel a mosdók többsége itt is üres mezővel vagy ‘Other’ értékkel rendelkezik, és az ürítési helyek legnagyobb hányada ilyen típusú mosdóknál van felsorolva.

	
	
		\newpage
			\section{Megerősítő adatelemzés}
	\subsection{Nemek szerinti vizsgálat}
	\paragraph{}
	\begin{compactlist}
		\item Az első hipotézis igazolásaként szűrtük az adathalmazt olyan esetekre, ahol van női, férfi és unisex mosdó is.	Ez a szűrés egy üres halmazt eredményezett, azaz egyáltalán nincs olyan mosdó, ahol van női, férfi és unisex mosdó egyszerre.
		\item A második hipotézis igazolásaként szűrtük az adathalmazt olyan esetekre, ahol csak női, vagy csak férfi mosdók vannak. az ilyen mosdók száma 117, ami az adathalmaz 0.6\%-a, ami valóban igen kevés eset.
	\end{compactlist}
	\subsection{Bevásárlőközpontok mosdói}
	\paragraph{}
		\Figref{toilets_in_shopping_centers}án csak a bevásárlóközpontokban lévő mosdókat ábrázoltuk. Ha összevetjük \figref{map}én lévő térképpel, akkor igazolódni látszik a hipotézis: a bevásárlóközpontokban lévő mosdók valóban a part közelében helyezkednek el. A már említett \textit{AliceSprings}ben lévő mosdó nincs a part közelében, azonban szokatlan elhelyezkedése miatt tekinthető outliernek.
		\begin{figure}[h]
			\centering	
			\includegraphics[scale=0.5]{in_shopping_centers}
			\caption{Bevásárlóközpontokban lévő mosdók földrajzi eloszlása}
			\label{fig:toilets_in_shopping_centers}
		\end{figure}		
		\begin{figure}[h]
			\centering	
			\includegraphics[scale=0.4]{australia_map}
			\caption{Ausztrália térképe (Google Maps)}
			\label{fig:map}
		\end{figure}

	\subsection{Mosdók sűrűsége}
	\paragraph{}
	Az utolsó hipozétis igazolásához szerettük volna a mosdók és a népesség eloszlását településenként összehasonlítani, azonban a népességre vonatkozó, megfelelő felbontású adathalmazt nem találtunk. Az általunk megtalált adatok a népesség államonkénti összesítését tartalmazza, így csupán ekkora felbontásban tudtuk összehasonlítani ezt a mosdók számával.\par
	\Figref{territories}án látható az egyes államokra jutó mosdók és lakosok száma, valamint az egy mosdóra jutó lakosok száma. Az adatokat táblázatos formában \tabref{territoires} tartalmazza. A mosdókról szóló adathalmaz összesen 18485 rekordot tartalmazott, míg Ausztrália teljes népessége 22,5 millió fő, így a teljes országot nézve az egy mosdóra 1163 lakos jut. Az államokra vonatkozó arányszámok kettő kivétellel $\pm$10\%-on belül mozognak. A két kivétel a főváros és Tasmania. Az előbbi eltérését a kiugróan nagy népsűrűséggel lehet megmagyarázni szerintünk, az utóbbit azonban nem tudtuk megmagyarázni széleskörű földrajzi és kulturális háttértudás hiányában. Egy szakértő valószínűleg rá tud mutatni néhány lehetséges okra. Mivel az arányszámok megközelítően egyenlőek a 8-ból 6 államban, így véleményünk szerint ezt a hipotézist is sikerült igazolni. 
		\begin{figure}[h]
			\centering	
			\includegraphics[scale=0.2]{territories}
			\caption{A mosdók és lakosok száma, valamint egy mosdóra jutó lakosok száma}
			\label{fig:territories}
		\end{figure}
		\begin{table}[h]
		\centering
		\begin{tabular}{ | l | r | r | r | }
			\hline
			\textbf{Állam neve} & \textbf{Lakosság (fő)} & \textbf{Mosdók száma (darab)} & \textbf{Lakos/mosdó (fő/darab)} \\ \hline
			Nyugat-Ausztrália & 2,23 millió & 2078 & 1077\\ \hline
			Északi-terület & 211945 & 199 & 1065\\ \hline
			Dél-Ausztrália & 1,6 millió & 1497 & 1066\\ \hline
			Queensland & 4,33 millió & 2078 & 1233\\ \hline
			Új-Dél-Wales & 6,92 millió & 6135 & 1127\\ \hline
			\textit{Főváros} & \textit{357222} & \textit{176} & \textit{2029}\\ \hline
			Victoria & 5,35 millió & 4200 & 1275\\ \hline
			\textit{Tasmania} & \textit{495354} & \textit{688} & \textit{719} \\ \hhline{|=|=|=|=|}
			\textbf{Összesen} & \textbf{22,5 millió} & \textbf{18485} & \textbf{1163} \\ \hline
		\end{tabular}
		\caption{A mosdók és lakosok száma, valamint egy mosdóra jutó lakosok száma szöveges}
		\label{tab:territoires}
		
	\end{table}\par
	
	\end{sloppypar}
\end{document}