		\section{Bevezetés}
		\subsection{Adatforrás}
		\paragraph{}	
		A házi feladatunk Ausztrália 18485 egységből álló, publikus használatban lévő mosdóinak az elemzése volt. Ehhez rendelkezésünkre állt egy állandóan frissülő .csv fájl, amely ezeknek a mosdóknak a különböző tulajdonságait tartalmazza.\par		
		Forrásaink a feladatkiosztásnál megjelölt \href{https://data.gov.au/dataset/national-public-toilet-map}{honlap}, illetve a mosdók állam által üzemeltetett közös \href{https://toiletmap.gov.au}{honlapja} volt. Az összefoglaló .csv fájlban a mosdók adatairól olvashattunk, míg az egyes attribútumok jelentését az egységes honlap segítségével társítottuk.
		\subsection{Attribútumok}
	\begin{compactlist}
		\item \textbf{Tulajdonság neve} - \textit{Típusa} : Jelentése (Megjegyzés, ha van)
		\item \textbf{Toilet ID} - \textit{Egész szám} : a mosdó azonosítója (1-től indul a számozás)
		\item \textbf{URL} - \textit{Szöveg} : a mosdó egyedi honlapja 
		\item \textbf{Name} - \textit{Szöveg} : a hely vagy intézmény neve, ahol a mosdó található
		\item \textbf{Address1} - \textit{Szöveg} : közterületi egység (sugárút/út/utca stb.)
		\item \textbf{Town} - \textit{Szöveg} : város
		\item \textbf{State} - \textit{Szöveg} : térség
		\item \textbf{Postcode} - \textit{Szöveg} : irányítószám
		\item \textbf{AddressNote} - \textit{Szöveg} : megjegyzések az elhelyezkedéssel kapcsolatosan (például a mosdó a park észak-nyugati csücskében van)
		\item \textbf{Male} - \textit{Logikai} : férfi mosdó
		\item \textbf{Female} - \textit{Logikai} : női mosdó
		\item \textbf{Unisex} - \textit{Logikai} : uniszex mosdó
		\item \textbf{DumpPoint} - \textit{Logikai} : ürítési pont (szemét, lakókocsiban felgyülemlett vizelet stb. elhelyezésére)
		\item \textbf{FacilityType} - \textit{Szöveg} : adódó lehetőségek, ahol a mosdó helyezkedik (például buszmegálló, park vagy sportpálya)
		\item \textbf{ToiletType} - \textit{Szöveg} : mosdó típusa (például szennyvízelvezetéses vagy sem)
		\item \textbf{AccessLimited} - \textit{Logikai} : korlátolt hozzáférés
		\item \textbf{PaymentRequired} - \textit{Logikai} : kell-e érte fizetni
		\item \textbf{KeyRequired} - \textit{Logikai} : kulcsot kell-e kérni
		\item \textbf{AccessNote} - \textit{Szöveg} : megjegyzések a hozzáféréssel kapcsolatosan (például a mosdó egy bevásárlóközpontban van)
		\item \textbf{Parking} - \textit{Logikai} : parkolási lehetőség
		\item \textbf{ParkingNote} - \textit{Szöveg} : parkolással kapcsolatos megjegyzések
		\item \textbf{AccessibleMale} - \textit{Logikai} : férfi mozgássérült mosdó
		\item \textbf{AccessibleFemale} - \textit{Logikai} : női mozgássérült mosdó
		\item \textbf{AccessibleUnisex} - \textit{Logikai} : uniszex mozgássérült mosdó
		\item \textbf{AccessibleNote} - \textit{Szöveg} : megjegyzés a mozgássérült mosdókkal kapcsolatban (például kerekes székkel csak az egyik irányból lehet megközelíteni)
		\item \textbf{MLAK} - \textit{Logikai} : MLAK kulcs használható-e (MLAK = Master Locksmiths' Association Key, egy mesterkulcs a publikus mosdók nyitásához, amelyet külön lehet igényelni orvosi engedéllyel)
		\item \textbf{ParkingAccessible} - \textit{Logikai} : mozgássérült parkoló lehetőség
		\item \textbf{AccessibleParkingNote} - \textit{Szöveg} : mozgássérült parkolási megjegyzések (például csak az utcafronton lehet parkolni)
		\item \textbf{Ambulant} - \textit{Logikai} : ambuláns betegeknek használható-e
		\item \textbf{RHTransfer} - \textit{Logikai} : jobbkezes nyitású ajtó mozgássérülteknek
		\item \textbf{LHTransfer} - \textit{Logikai} : balkezes nyitású ajtó mozgássérülteknek
		\item \textbf{AdultChange} - \textit{Logikai} : felnőtt pelenkázó
		\item \textbf{OpeningHoursSchedule} - \textit{Szöveg} : nyitvatartási idő
		\item \textbf{OpeningHoursNote} - \textit{Szöveg} : nyitvartartáshoz megjegyzések
		\item \textbf{BabyChange} - \textit{Logikai} : gyermek pelenkázó
		\item \textbf{Showers} - \textit{Logikai} : zuhanyzó
		\item \textbf{DrinkingWater} - \textit{Logikai} : ivóvíz lehetősége
		\item \textbf{SharpsDisposal} - \textit{Logikai} : gyógyszer- és vegyi anyag tároló
		\item \textbf{SanitaryDisposal} - \textit{Logikai} : egészségügyi hulladék tároló
		\item \textbf{IconURL} - \textit{Szöveg} : jelölő ikon URL-je
		\item \textbf{IconAltText} - \textit{Szöveg} : szövegezés (milyen mosdólehetőségek vannak, röviden, például női-férfi-uniszex)
		\item \textbf{Notes} - \textit{Szöveg} : egyéb megjegyzések (például van a közelben BBQ étterem és lehetőség)
		\item \textbf{Status} - \textit{Szöveg} : Hitelesített vagy sem (Verified/Unverified) (hitelesített vagy nem)
		\item \textbf{Latitude} - \textit{Lebegőpontos szám} : szélességi fok
		\item \textbf{Longitude} - \textit{Lebegőpontos szám} : hosszúsági fok
	\end{compactlist}
	\paragraph{}
	Néhány szöveges mezőhöz valójában kategorikus változónak feleltethető meg. A két legfontosabb ezek közül a \textit{FacilityType} és a \textit{ToiletType}. A \textit{FacilityType} a következő értékeket veszi fel:
	\begin{compactlist}
		\item Airport
		\item Bus station
		\item Camping ground
		\item Car park
		\item Caravan park
		\item Food outlet
		\item Other
		\item Park or reserve
		\item Service station
		\item Shopping centre
		\item Sporting facility
		\item Train station
		\item Üres
	\end{compactlist}
	A \textit{ToiletType} pedig az alábbiakat:
	\begin{compactlist}
		\item Automatic
		\item Compost
		\item Drop toilet
		\item Pit
		\item Sealed Vault
		\item Septic
		\item Sewerage
		\item Üres
	\end{compactlist}
	